	\section{Introduction:\\Metropolis algorithm and optimization}
		
\begin{frame}{Why and How?}
	\begin{center}
		From ``classical'' to ``ab-initio'' molecular dynamics:\\
		wider class of problems (``chemically complex'' systems)
	\end{center}
	Basic assumption:
	\begin{center}
		\alert{Born-Oppenheimer approximation}
	\end{center}
	A possible conceptual scheme:
	\begin{enumerate}[$i)$]
		\item set the initial configuration;
		\item minimization of the energy functional $E[\{\psi_i\},\,\{\textbf{R}_I\}]$ (\emph{BO surface});\\
		\item forces on ions via Hellmann-Feynman theorem;
		\item integration of ions' equations of motion and back to \structure{$ii)$}.
	\end{enumerate}
	\begin{center}
		\alert{Impracticable} scheme with \emph{DFT} or \emph{HF} methods
	\end{center}
\end{frame}


\section{The Car-Parrinello algorithm}

\begin{frame}{A unified approach}
	Fictitious electronic dynamics and proper ionic dynamics
	\begin{center}
		\alert{run in parallel!}
	\end{center}
	Formally:
	\[
		L = T - V
	\]
	where
	\begin{align*}
		T &= \frac{1}{2}\,\mu\,\int d^3r\,\absq{\dot{\psi}_i} + \frac{1}{2}\sum_I M_I\,{\dot{R}_I}^2 = \text{\emph{FKE}} + \text{\emph{IKE}} \\
		V &= E[\{\psi_i\},\,\{\textbf{R}_I\}] = E_\tu{el} + E_\tu{II}
	\end{align*}
	Electronic \emph{d.o.f.} constrained to be \emph{orthogonal}
	\[
		\int d^3r\,\psi_i^*\,\psi_j - \delta_{ij} = 0
	\]
\end{frame}
\begin{frame}%{A unified approach \ (2)}
	Choosing a convenient basis we solve for the coefficients:
	\[
		{\{\chi_k\}}_k\ \ \text{basis set} \quad \Rightarrow \quad \psi_i = \sum_k\,c_k^i\,\chi_k
	\]
	Then
	\begin{align*}
		T &= \frac{1}{2}\,\mu\,\sum_{ik} \absq{\dot{c_k^i}} + \frac{1}{2}\sum_I M_I\,\dot{\textbf{R}}_I \\
		V &= E(\{c_k^i\},\,\{\textbf{R}_I\}) \mapsto \tilde{V} = V + \sum_{ij}\,\Lambda_{ij}\left(\sum_k {c_k^i}^* c_k^j - \delta_{ij} \right)
	\end{align*}
	\structure{Euler-Lagrange equations} (with \structure{holonomic constraints}) are
		\[
			\begin{system}
				\mu\,\ddot{c_k^i} &= - \pd{E}{c_k^i} - \sum_j {\Lambda'}_{ij}\,c_k^j \\
				M_I\,\ddot{\textbf{R}}_I &= - \pd{E}{\textbf{R}_I}
			\end{system}
			\qquad \text{where \ }
			{\Lambda'}_{ij} =
				\begin{sys_cond}
					\Lambda_{ij}	&	\text{if }\ i\neq j \\
					2\,\re{\Lambda_{ii}}	&	\text{if }\ i = j
				\end{sys_cond}
		\]
	\begin{center}
		\alert{Energy-conserving dynamics}: $E_\tu{fictitious} + E_\tu{real} = \tu{constant}$
	\end{center}
\end{frame}


\begin{frame}{How can this work?}
	\begin{center}
		\emph{BO} surface condition $\qquad \Rightarrow \qquad$ \alert{$\mu$ very small!}
			(\emph{FKE} $\ll \abs{E_\tu{real}}$)\\
			\emph{CP} Lagrangian gives a \alert{quasimicrocanonical system}
	\end{center}
	\begin{columns}[h]
		\column{.4\textwidth}
			\begin{flushleft}
				\structure{\emph{CP}} (dynamic)\\
				Small oscillations on top of the \emph{BO} surface \\
				\alert{but}
				algorithm is \textbf{stable} for long times
				and gives \textbf{adiabaticity}
			\end{flushleft}
		\column{.1\textwidth}
			\center\structure{\sl vs}
		\column{.4\textwidth}
			\begin{flushright}
				\structure{SD + BO} (parametric)\\
				Much better convergence on the \emph{BO} surface
				\alert{but}
				system steadily loses energy
			\end{flushright}
	\end{columns}
	\vspace*{4ex}
	\begin{center}
		\alert{It is not a matter of convergence!}
	\end{center}
	Answer is in
	\begin{center}
		the \alert{averaging method} and the existence of \alert{adiabatic invariants}
	\end{center}
\end{frame}


\begin{frame}{A theoretical justification\ldots}%\framesubtitle{Why is the \emph{CP} Lagrangian so good?}
	\structure{Averaging methods}\\[3ex]
		\hspace*{.2\textwidth}
		\begin{minipage}[r]{.8\textwidth}
			Classical Hamiltonian $H = H_\tu{lf} + H_\tu{hf}$ approximated by $\langle H \rangle_\tu{h}$\\
			(average on the high frequency part)\\
			$\Rightarrow \qquad$ \emph{BO} approximation holds
		\end{minipage}\\[4ex]

	\structure{Adiabatic invariants}\\[3ex]
		\hspace*{.2\textwidth}
		\begin{minipage}[r]{.8\textwidth}
			Slow perturbation with frequency $\nu$, then there are quantities constants over times $\propto \nu^{-r}$ ($r \gtrsim 1/2$)\\
			$\Rightarrow \qquad$ adiabaticity over long times
		\end{minipage}\\[4ex]
	
	\begin{center}
		\alert{Empirically}: adiabaticity connected to the \alert{energy gaps in electronic spectrum}\\
		\alert{Formally}: good description provided by \alert{$n$-level models} ($n = 3$ accounts for almost all the physics)
	\end{center}
\end{frame}

\begin{frame}{\ldots and an explicative result}\framesubtitle{8 Si atoms, diamond structure}
	\begin{minipage}[c]{0.45\textwidth}
		\centering
		\includegraphics[width=.85\textwidth]{PSD}
	\end{minipage}
	\hspace*{.05\textwidth}
	\begin{minipage}[c]{0.45\textwidth}
		\centering
		\includegraphics[width=.85\textwidth]{energy_evol}
	\end{minipage}\\
	\begin{minipage}[c]{0.45\textwidth}
		\centering
		\includegraphics[width=.85\textwidth]{comp_BO_CP}
	\end{minipage}
	\hspace*{.05\textwidth}
	\begin{minipage}[c]{0.45\textwidth}
		\centering
		\includegraphics[width=.85\textwidth]{diff_BO_CP}
	\end{minipage}
\end{frame}


\section{Implementation}

\begin{frame}{Energy density-functional (\emph{DFT} and \emph{KS})}
	Recall the \alert{Hohenberg-Kohn theorem}: ground state completely determined by its density.\\
	One writes the energy as a \alert{density-functional}:
	\[
		E[n(\textbf{r})] = \underbrace{\frac{1}{2}\sum_i\int d^3r\, \psi_i^*(-\grad^2)\psi_i}_{T[n]} + \underbrace{\iint d^3r\,\frac{n(\textbf{r})\,n(\textbf{r}')}{|\textbf{r} - \textbf{r}'|}}_{E_{Hartree}} + E_\tu{XC}[n] + E_\tu{ext}
	\]
	Minimization with respect to $n(\textbf{r})$ gives the \alert{Kohn-Sham equations} (for the best single-particle orbitals)
	\[
		\left\{ -\frac{1}{2}\,\grad^2 + U_\tu{Hartree}(\textbf{r}) + \mu_\tu{XC}(n(\textbf{r})) + v(\textbf{r}) \right\}\psi_i = \varepsilon_i\,\psi_i
	\]
\end{frame}

\begin{frame}{Algorithm}
	\begin{itemize}
		\item Assumption of periodicity: convenient to use \alert{plane waves}\\
			Use of \alert{pseudopotentials} allows a small basis set
		\item \alert{\emph{FFT} methods} to compute ``forces'' on the electronic coefficients: $\partial E/\partial c_k^i$
		\item \alert{Velocity Verlet / \emph{SHAKE}} to integrate the \emph{CP} equations with constraints\\
			(\alert{Nosè-Hoover thermostats} if electronic spectrum with small energy gaps)
	\end{itemize}	
\end{frame}


%\begin{frame}{Domande}
%	\begin{itemize}
%		\item perché pseudopotenziali non locali?
%		\item \ldots
%	\end{itemize}
%\end{frame}